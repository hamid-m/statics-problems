
\documentclass{article}

\usepackage{amsmath,amsfonts,epsfig,subfigure,import}


% Change Margins: This gives margins with 1in on top, 
% .75 in on bottom and .75in on each side.
\textheight 9.25in
\textwidth 7in
\topmargin -.57in 
\oddsidemargin -.25in 
\evensidemargin -.25in

%\setlength{\evensidemargin}{0in}
%\setlength{\oddsidemargin}{0in}
%\setlength{\textwidth}{6.5in}
%\setlength{\textheight}{9in}

\pagestyle{empty}

\begin{document}

% Set up commands 

\vspace{.5cm}
\noindent
\textbf{Name: }\rule{5cm}{0.4pt}\hspace{.5cm}
\textbf{Section: }\rule{1cm}{0.4pt}
%\hfill\textbf{AEM 2011 Quiz \#1}
%\hfill\textbf{AEM 2011 Quiz \#2}
%\hfill\textbf{AEM 2011 Quiz \#3}
\hfill\textbf{AEM 2011 Quiz \#4}

\noindent
%\hfill Tuesday, January 24, 2023
%\hfill Tuesday, January 31, 2023
%\hfill Tuesday, February 7, 2023
\hfill Tuesday, February 14, 2023
\vspace{1cm}
\noindent

\vspace{.3cm}

\noindent A non-communicating calculator is allowed.  Full credit will only be given if all steps used are clearly communicated (free body diagrams, algebra, etc).

\vspace{0.5cm}

% Briefly justify your answers.
% QZ1
%\noindent\import{probs/hanging-robot}{prob-qz.tex} 

% QZ2
%\noindent\import{probs/02.5-skycam}{prob-qz.tex} 

% QZ3
%\noindent\import{probs/03.1-tightrope}{prob-qz.tex} 

% QZ4
\noindent\import{probs/03b-BJM-12ed}{prob-qz.tex} 
\end{document}

